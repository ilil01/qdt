\documentclass[conference,compsoc,a4paper]{IEEEtran}
\IEEEoverridecommandlockouts

\usepackage{multirow}

\ifCLASSOPTIONcompsoc
  % IEEE Computer Society needs nocompress option
  % requires cite.sty v4.0 or later (November 2003)
  \usepackage[nocompress]{cite}
\else
  % normal IEEE
  \usepackage{cite}
\fi

\hyphenation{op-tical net-works semi-conduc-tor}

\begin{document}

\title{Peripheral device register support for\\%
source code boilerplate generator of\\%
QEMU Development Toolkit}


% author names and affiliations
\author{
    \IEEEauthorblockN{
        Vasily Efimov\IEEEauthorrefmark{1},
        Vartan Padaryan\IEEEauthorrefmark{1}\IEEEauthorrefmark{2}}
    \IEEEauthorblockA{\IEEEauthorrefmark{1}
        Ivannikov Institute for System Programming of the RAS,
        ul. Solzhenitsyna 25, Moscow, 109004 Russia}
    \IEEEauthorblockA{\IEEEauthorrefmark{2}
        Moscow State University, Moscow, 119991 Russia
        Email: \{real, vartan\}@ispras.ru}
}


\maketitle

\begin{abstract}
QEMU emulator is used together with different security tools to solve problems
where maximum control over virtual machine is required.
Absence of a proper VM with specific devices is a widespread issue.
Also, development of a new virtual device (device model) is a error prone and
time consuming process.
The QEMU Development Toolkit (QDT) can be  used to simplify the first
stage of this process by generating boilerplate code for models.
Currently QDT can generate the bulk of device model code.
The paper discusses some extensions of QDT capabilities.
A device register model was added.
The register model allows to generate extra code for frequently met
device register behavior patterns.
The model is implemented as an  extension to QDT device description
API in Python.
Evaluation showed that size of device register description is half of
C-code boilerplate generated from it.
The register description is gathered in one place, that's more convenient
for developer whereas generated code is spread across the boilerplate.
\end{abstract}

\begin{IEEEkeywords}
QEMU, boilerplate generation, device semantic.
\end{IEEEkeywords}


\IEEEpeerreviewmaketitle


\section{Introduction}
The QEMU \cite{QEMU} emulator is used together with different security tools
for problems where
maximum control over virtual hardware is required, such as malware reverse
engineering,
vulnerability searching \cite{QEMUIdDynAn} and so on.
While there had been efforts to improve emulator infrastructure
\cite{QEMU-min-replay},
in practice there is lack of specialized virtual platforms.
For example, the diversity of the IoT leads to the situation that the security
analysis of every new device will be preceded by development of an appropriate
VM.
Such a situation unacceptably delays practical results, because existing
practices of development of a new VM for QEMU last up to a year.
The QEMU Development Toolkit (QDT) \cite{QDT} had been proposed to simplify and
speed up the VM development.
The initial version could generate the bulk of code of a device model
boilerplate.
This paper discusses additional features that were implemented in QDT; a device
register model was added.
The register model extends capabilities of the device register bank boilerplate
generator adding extra code for implementation of frequently encountered device
register behaviour.

This paper first presents a brief overview of the way QEMU models devices.
It is followed by an explanation of the device boilerplate generator
capabilities.
Then, the proposed device register model is explained.

\section{Devices in QEMU}

This section describes both QEMU approach to device modelling and the tricks
QDT uses to generate device source code boilerplates.
In course of it the aspects of developer job improved by this work are
designated.

A device in QEMU is a set of functions (handlers) and structures initialized
with those functions addresses and related data.
QEMU infrastructure uses this information to bring the device to operation.
Each function serves its specific purpose.
In other words, device model is event driven.

Given a developer preferences, QDT generates those functions and structures.
Device boilerplate generation is based on \textit{templates} (do not confuse
with C++ templates).
A template is a parametrized format string.
But frequently a device feature is a set of templates with a complicated
interrelation.
Interrelations may be between different features too.
Those of this kind are normally much complicated than feature inner
interrelations.
A code in C language is produced by assigning values to template parameters.

Unfortunately, there is no advanced semantic related preferences.
Hence, function bodies are often quite empty and structures are filled with
example values.

This work makes a step towards this problem by introducing the device register
model which is actually some kind of device semantic description.
By using of this register information QDT generates much more code than
originally.

\subsection{Device operation}

The life cycle of a basic QEMU virtual device consists of four stages:
instantiation, resetting, normal operation and destruction.

\subsubsection{Instantiation and destruction}
Instantiation is split onto initialization and realization.
Symmetrically, destruction is split onto \textit{un}realization and finalization.

During initialization the handler should configure device operation aspects
whose do depend neither on external conditions nor runtime.
A good example is \textit{properties} declaration.
A property services for configuration of the device by machine instantiation
code.
Examples of property purposes are:

\begin{itemize}
    \item file-image (block device namely) assignment for a ROM device,
    \item NIC connection to an emulated hub (which is normally connected
    to either TAP or a host NIC at other port),
    \item UART (serial interface) connection to a virtual terminal
    emulator, file, network socket etc.,
    \item device specific operation mode setting and so on.
\end{itemize}

It is guaranteed that no property is changed after device is realized.
In other words, all property initializations are made between calls of the
initialization handler and the realization handler.
Except properties, there are other examples of initilization handler code.
They will be given below during explanation of other device parts.

Realization handler should made all rest work excluding resetting of
runtime device parameters.
It is responsibility of reset handler.
Normally, realization handler contains code that depends on property
values.

Unrealization handler removes device references to other board components
and finalization handler frees extra heap memory.
Note that the memory block allocated for the device itself is freed by
QEMU infrastructure.

Initially, the bodies of initialization, realization, unrealization and
finalization handlers are empty.
But there are many other templates that require a code to be executed
during those handlers.
As a result, a generated content of them are normally far from few lines.

\subsubsection{Resetting}

Reset handler brings the device to an initial state according to
its documentation.
Many boards support runtime reset by both hardware and software causes.
So, this handler is often called several times.

Some device features need to be reseted and corresponding templates do
add its specific code to the handler.
Going forward, because of device registers must be reseted the feature
added by this work increases size of reset handler boilerplate much.

\subsubsection{Normal operation}

There is no explicit handler for normal operation.
Instead there are many handlers declared during instantiation.
QEMU infrastructure calls them when needed.
Each of those handlers corresponds to a device feature.

\subsection{Device features}

A device model is a composition of features whose all together emulate
device behaviour.
The features supported by QDT boilerplate generator is described below.

\subsubsection{State structure}

The state of a device is held into a structure.
The behaviour of all device features depends on current state.
Hence, the content of this structure is touched by almost all
the feature templates.
QDT has a simple template of state structure and other templates
do fill it during generation.

\subsubsection{State snapshot}

Some fields of state structure are changed dynamically during normal
operation.
Because QEMU supports snapshot mechanism, the device model should
declare such fields in the state description variable and pass it
to the emulator during instantiation.
QDT meets this requirements using an empty template, which is
filled by other feature templates.

\subsubsection{Registers}

Registers are used by a device to exchange data with other machine
components.
For instance, those components are CPU and DMA controller.
Registers are frequently joined in banks.
A register bank cowers a segment of the device bus address space.
Name of register banks are differing from one bus standard to another.
For example, registers of system bus are called ``MMIO''
(memory mapped input/output).
Legacy LPC registers of IBM PC architecture is called ``PIO''
(programmable input/output).
PCI registers have name ``BAR'' (base address register).
Whatever bus, QEMU uses the same approach for register bank
implementation.

A register bank is defined by both read and write handlers.
Any time a something in the system accessed the register bank,
the corresponding handler is called.
Except handlers, there is a structure and a code in
instantiation handlers.
QDT does generate boilerplates for all this items.

The handler body is usually a big \texttt{switch} block.
Each \texttt{case} of which corresponds to a register of the bank.
Originally, QDT generated only \texttt{default} statement which prints
the message about an unimplemented access.
Also, registers must have corresponding fields in both state structure
and state declaration.

Before this work, a developer have to do the tedious job of
filling this all by hands.
This is because there were no semantic information about a register
bank.
But some regular signatures of registers were discovered and the register
model was developed.
It is described in details in the separated section below.

\subsubsection{Interrupt requests}

IRQ is a way device uses to notify software about some event.
Some devices (like an interrupt controller or a bus bridge) have input
IRQs too.
Input IRQ does require a handler.
Alos using of this feature extends instantiation handlers and the
state structure.
QDT provides this automatically.

During emulation of IRQ behaviour the register state is frequently
affected from within handler of any related feature.
This is implemented by developer hands.
Therefore, it is very handy when the register already have a field in the
state structure.

\subsubsection{Timers}

In a real device a timer is a register that is being decremented or
incremented each tick of the clock.
When the register reaches some value an event happened.
But QEMU does not emulates clock ticking so precise.
Instead a device model should evaluate the time at which this event will
took place and schedule a handler to be called.
This handler does execute required actions.

In other words, a timer in QEMU is a callback function with specific
descriptor.
Normally, timers are used to implement IRQ raising behaviour.
Of course, some registers are involved in the process.

QDT does generate a simple handler, auxiliary code in both instantiation
and destruction handlers, and fields in both state description and
state structure.
But implementation of register value accounting is up to a developer.
So, registers are desired to be implemented already.

\subsubsection{Other features}

There are other features supported by the QDT device boilerplate generator
including:

\begin{itemize}
    \item character device (used to implement serial interfaces),
    \item block device (used to implement ROMs),
    \item network interface abstraction.
\end{itemize}

The registers are used by all of them.
And the goal of this work is to provide those registers in the boilerplate
before future development started.

\section{Device registers}

Basing on experience of device models development a several frequently
encountered register behaviour signatures were discovered.
The formal model was invented.
This model does not covers all aspects of register semantic.
Instead it is used to automatically generate some frequent patterns.

The model is implemented as an optional extension to device description
of QDT.
Recall that QDT is written in Python and uses this language for
configuration files too.
In other words, this work extends API which QDT used for device
boilerplate description.

The register model provides formal way for register bank definition.
Register bank is given by a list of register descriptors.
Order of the descriptors is significant because of it is used to build
\texttt{switch} block of handlers.
Each descriptor contains information given below.

\begin{itemize}
    \item \textit{Short name} is used to evaluate a name for state structure
    field.
    \item \textit{Long name} is used in automatically generated comments for
    a developer.
    \item \textit{Size} is used to choose appropriate type
    for state structure field, macro for state description record,
    and to build \texttt{switch} blocks of handlers.
    \item \textit{Initial value} is loaded into the register during device
    resetting.
\end{itemize}

Other information held in a register descriptor configures semantic.
There are four semantic classes used in this work: read only registers,
read/write registers, write only registers and gap registers.

\subsection{Read only register}

Writing to such register takes no effect on the device.
Register of this class is normally used for storing data the device
prepared for CPU.
An example is RX register of a UART which stores data received by
the UART from channel.

A code of the read helper is as simple as structure field reading.
A message is added to the write handler to warn a developer about
unexpected access.

\subsection{Read/write register}

This kind of registers is normally used for device configuration.
Content of a configuration register is normally handled by the device
as input.

In basic case a code of both read and write handlers is actually a
structure field access.
But sometimes several bits of a read/write register has access
restrictions.
There are three known limitation kinds: read only bit,
write after read (WAR) bit and write once bit.
First two of them are implemented.
An explanation is below.

\subsubsection{Read only bit}

Read only bits are usually used for reserved values, error or event
flags, hardware provided signals and so on.
To denote those bits a bitmask is used.
If a register has such bits the write handler is extended with
the expression using this bitmask to protect read only bits from
being overwritten.

\subsubsection{Write after read bit}

Write after read bits are only allowed to be written if they were
read since last changing by the device.
A bit of such kind is frequently used as IRQ cause status flag.
This prevents loosing of the IRQ cause by the software.
An example is presented below.

Let a device has one IRQ which is raised by two events.
For simplicity, let names of the events are ``A'' and ``B''.
Status bits have correspondingly same names.
Both bits are in same status register.
The sequence leading to losing of event notification is here.

\begin{enumerate}
    \item Event ``A'' happened.
    The bit ``A'' is set and the IRQ is raised.
    \item Guest software reads status register to get the reason of the
    IRQ.
    \item Event ``B'' happened.
    The bit ``B'' is set and the IRQ is still raised.
    \item The software handled event ``A''.
    Then it resets bit ``A'' in cached value and writes it back to notify
    the device that event ``A'' was handled.
    Note that bit ``B'' is \textit{not} set in \textit{cached} value.
    Software expected the device lowers IRQ now.
    Also reminder that a software must cache register value and affect only
    specific bit because there are different important information in
    neighboured bits.
    But a software may access memory by machine words.
    And a word do fit entire register.
    In other words, a software may not write to a specific bit
    independently.
    Therefore, this software behaviour is normal.
    \item Without write after read protection the bit ``B''
    is overwritten with the cached value (before event ``B'') and
    the software will never know about event ``B''.
\end{enumerate}

User of QDT designates WAR bits of a register by a bitmask.
Implementation of those bits in a device model is based on extra
field in the state structure.
This field stores a bitmap.
If a bit in the bitmap is reset (zero) then the corresponding bit of
the register is write protected.
That means, the same approach as for read only bits is used.
The difference is the ``write mask'' in this case is dynamic.
Of course, the bitmap is declared in the state description to be saved
in a snapshot.

The implementation described above is provided by an auxiliary code
which is generated automatically by QDT.
QDT is also provides a code in the read helper which does set all
bits in the bitmap making all WAR bits writable.
But QDT does not know when those bits are set by the device.
So, resetting of corresponding bits in the bitmask is up to developer.

\subsubsection{Write once bit}

Write once bit becomes write protected after first writing it.
A condition that makes this bit writable again may be arbitrary.
Frequently, such bits become writable on the device reset.

The example of write once bits (entire register actually) is
subsystem and subsystem vendor identifiers of PCI device
configuration space.
They are set by PCI ROM at BIOS boot time and never changed by
software during system operation.

Implementation of write once bits is beyond this work because of
there no known examples of such bits except PCI.
But write once bit support for it is already implemented in QEMU.

\subsection{Gap register}

This is not a register.
It is actually a gap between registers.
Writing to it takes no effect.
Reading of it returns the value defined by the documentation for such
cases.
Frequent examples of the value is either all zeros (0x0...0) or all
ones (0xF...F).
This class of ``registers'' is supported by the model to allow a developer
to avoid splitting a register bank onto two ones.
It is handy under some circumstances.

\subsection{Write only register}

There are no known examples of write only registers.
Bit it still technically possible.
The implementation of such register is same as implementation of read/write
register with one exception: reading of it results in a warn message.
The value returned by read helper is normally same as for gap register.

\section{Evaluation}

Evaluation of introduced QDT extension is based on comparison of sizes of
originally resulting device boilerplate and the boilerplate of same device with
register description.
Line count is used as size measure.
Measurements are made for several peripheral devices of different
purposes.
Exact device model names cannot be revealed in this paper.
Alias names will be used instead.

The comparison is given in Table 1.
There are several notes about its content interpretation to be given.

\begin{itemize}
    \item A line of device model description for QDT is as simple as
    keyword argument assignment.
    \item A line of boilerplate is an arbitrary C code line.
    That means, it can be an empty indentation line or a complicated
    expression or long function declaration etc.
    \item A device boilerplate is normally presented by module and header.
    It also registered in corresponding Makefile.
    There is sum of lines of all those files in the table.
\end{itemize}

% Shortcuts for table
\def \r#1{\multirow{ 2}{*}{#1}}

\begin{table}[!t]
% increase table row spacing, adjust to taste
\renewcommand{\arraystretch}{1.3}
\caption{Impact of Device Register Model Appliance}
\label{table_example}
\centering
\begin{tabular}{|c||c|c||c|c|}
\hline
Device     & \multicolumn{2}{c||}{Description size} & \multicolumn{2}{c|}{Boilerplate size}\\
\cline{2-5}
name       & without    & with       & without   & with \\
           & registers  & registers  & registers & registers\\
\hline
UART       & 7          & 54         & 157       & 247\\
Interrupt  & \r{7}      & \r{40}     & \r{127}   & \r{196}\\
Controller &            &            &           &\\
Timer      & 7          & 120        & 152       & 375\\
Watchdog   & 7          & 32         & 133       & 177\\
ROM        & 6          & 54         & 130       & 224\\
\hline
\end{tabular}
\end{table}

As previously, a resulting device boilerplate with registers is not
ready to use because there is no all the semantic.
But it can be compiled and provides a start point for consequent
development.

\section{Conclusion}

The device register model introduced by this work simplifies register
bank implementation.
The evaluation showed that size of device register description is half of
C boilerplate generated from it.
It is also important that the register description is congregated in one
place while corresponding generated code is spread across the C boilerplate.
Hence, it is more easy to configure registers in the description.
Moreover, the previous register bank implementation way is error prone
because there are many related places across both the module and the header.


\ifCLASSOPTIONcompsoc
  \section*{Acknowledgments}
\else
  \section*{Acknowledgment}
\fi

QDT development is originally supported by RFBR, grant № 16-29-09632.
This work was made as part of ISP RAS work flow and supported indirectly
by corresponding projects.

\begin{thebibliography}{1}

\bibitem{QEMU}
F.~Bellard QEMU, A Fast and Portable Dynamic Translator. USENIX Annual
Technical Conference, FREENIX Track. USENIX, 2007. P. 41-465~p.

\bibitem{QEMUIdDynAn}
Dovgalyuk~P.M., Makarov~.A., Padaryan~V.A., Romaneev~M.S.,
Fursova~N.I. Primenenie programmnykh ehmulyatorov v zadachakh
analiza binarnogo koda [Application of software emulators for
the binary code analysis]. Trudy ISP RAN [The Proceedings of
ISP RAS], 2014, vol.~26, no.~1, pp.~277-296.

\bibitem{QEMU-min-replay}
V.~Efimov, K.~Batuzov, V.~Padaryan, A.~Avetisyan. Features of the deterministic
replay in the case of a minimum device set. Programming and Computer
Software, vol. 42, pp.174-186, Apr 2016

\bibitem{QDT}
V.Yu.~Efimov, A.A.~Bezzubikov, D.A.~Bogomolov, O.V.~Goremykin,
V.A.~Padaryan, “Automation of device and machine development
for QEMU”, Proceedings of ISP RAS, 29:6~(2017), 77–104

\end{thebibliography}


\end{document}

